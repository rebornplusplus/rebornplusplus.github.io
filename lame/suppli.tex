\documentclass[]{article}
\usepackage[utf8]{inputenc}

\title{On Supplementary Exams}
\author{Rafid Bin Mostofa}
\date{March 13, 2020}

\begin{document}
	\maketitle
	\hrule \vspace{0.5cm}

	\noindent I won't shit you with why you shouldn't fail. You know better.
	
	\section{Registration}
	First off, you need to register for a suppli at the very beginning of the term. Normally registration opens up at the very first week of a term and stays open for maybe a week or so. But be careful, because it doesn't stay open for long. I will walk you the steps for a regular suppli since I didn't participate in any graduate suppli. Also this is as of March 2020. Things might not be the same. \vspace{0.5cm}

	Let's get down to the steps:
	\begin{enumerate}
		\item \textit{Clear your dues.} \\ If you have any dues and want to sit for a suppli next term, clear your dues beforehand. With dues remaining, you'd have lots of pain to follow some already not so very simple steps.
		\item \textit{Collect gradesheet of last term.} \\ Collect your last term gradesheet. Normally they wind up at the department office automatically. However, if they don't (mostly because of remaining dues, which they term ``Comptroller''), you'd have to go all the way to Exam Controller, most likely with some receipt that your dues are cleared.
		\item \textit{Collect suppli form.} \\ You can find them in your departmental office. Fill it out. There should be some tabs for ``Credit hours completed'' and ``Credit hours due''. You can find the former in your last term gradesheet. The latter is the sum of lags in credit hours.
		\item \textit{Get a forwarding from your advisor.} \\ Meet your advisor. Get his forwarding. There should be a dedicated box at the form. Don't ask him about steps to follow. Most likely, he doesn't know.
		\item \textit{Pay the bank.} \\ Normally, I pay the bank after I get a forwarding, but if your advisor suggests to pay the bank first, do it. As of now, to sit for a $x$ credit suppli, you need to pay $x \cdot 500$ in BDT. Grab a form from the registrar office. They should be right outside the Comptroller office, first floor corridor to be exact. Fill it out, with the amount of money in the tab for ``Exam fees''. Submit it in the bank, get your copy.
		\item \textit{Submit the form.} \\ If things go alright so far, you have a copy of your last term gradesheet and a bank copy of payment. Attach photocopy of both of these with your suppli form and submit them in your departmental office.
		\item \textit{Register in BIIS.} \\ Normally my department calls me after a day or two, to confirm the application. If your doesn't, well, sad for you. In either cases, go to BIIS, then ``Register (Supplementary)'', choose ``General''. You should have a list for your applied courses (in this case, 1). Tick those and submit registration. You should be given a registration id. Save it. Also, you should mail your advisor about your registration in BIIS. Last time, I didn't. But worked.
	\end{enumerate}

	I hope those worked. Here are a few pointers if they didn't:
	\begin{itemize}
		\item Should you not be able to find the course list in BIIS to register, contact your departmental office immediately. If they confirm the approval of your application, but still you are not getting it, complain in BIIS office. Seventh floor, first door to the left of lifts.
		\item If, somehow, you are unable to get the ``Provisional Grade Sheet'', print out your result from BIIS and submit it. Worked for me a couple times.
		\item Normally suppli courses are approved, but be sure to check it first. Usually, they don't refuse an application. So everyone pays the bank prior to submitting the form. However, talk to your advisor if your desired course is no more.
	\end{itemize}

	\section{Exam}
	They provide a notice in BUET's official site mentioning the probable dates for the supplimentaries. They typical schedule is in the mid. However, if they don't happen in the mid vacation, most of the times they happen the next week. Keep an eye out for this.

	\section{Result}
	Within an week or two, you should receive your grades for the suppli in BIIS. If you didn't know already, your max GPA would be a 3.00.

\end{document}